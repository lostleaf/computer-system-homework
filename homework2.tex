%%%%%%%%%%%%%%%%%%%%%%%%%%%%%%%%%%%%%%%%%%%%%%%%%%%%%%%%%%%%%%%%%%%%%%%%
\documentclass[12pt]{article}
\usepackage{amssymb}
\usepackage{amsmath,amssymb,CJK}
\usepackage{graphicx}
\usepackage{subfigure}
\usepackage{listings}
\usepackage{enumerate}


\openup 7pt\pagestyle{plain} \topmargin -40pt \textwidth
14.5cm\textheight 21.5cm
\parskip .09 truein
\baselineskip 4pt\lineskip 4pt \setcounter{page}{1}
\def\a{\alpha}\def\b{\beta}\def\d{\delta}\def\D{\Delta}\def\fs{\footnotesize}
\def\g{\gamma}
\def\G{\Gamma}\def\l{\lambda}\def\L{\Lambda}\def\o{\omiga}\def\p{\psi}
\def\se{\subseteq}\def\seq{\subseteq}\def\Si{\Sigma}\def\si{\sigma}\def\vp{\varphi}\def\es{\varepsilon}
\def\sc{\scriptstyle}\def\ssc{\scriptscriptstyle}\def\dis{\displaystyle}
\def\cl{\centerline}\def\ll{\leftline}\def\rl{\rightline}\def\nl{\newline}
\def\ol{\overline}\def\ul{\underline}\def\wt{\widetilde}\def\wh{\widehat}
\def\rar{\rightarrow}\def\Rar{\Rightarrow}\def\lar{\leftarrow}
\def\Lar{\Leftarrow}\def\Rla{\rightleftarrow}\def\bs{\backslash}
\def\ra{\rangle}\def\la{\langle}\def\hs{\hspace*}\def\rb{\raisebox}
\def\ni{\noindent}\def\hi{\hangindent}\def\ha{\hangafter}
\def\vs{\vspace*}
\def\hom#1{\mbox{\rm Hom($#1,#1$)}}
\def\thebeg{\vskip8pt\par\ni}
\def\theend{\vskip5pt\par}
\def\chabeg{\pagebreak\par}
\def\chaend{\vskip20pt\par}
\def\secbeg{\vskip15pt\par}
\def\secend{\vskip10pt\par}
\def\exebeg{\vskip10pt}
\def\exeend{\vskip6pt}
\def\undot#1{\mbox{$\mbox{#1}\hs{-1.5ex}_{_{\dis\centerdot}}\,\,$}}
\def\qed{\hfill\mbox{$\Box$}}
\def\C{\mathbb{C}}
\def\Q{\mathbb{Q}}
\def\ii{\mbox{\,{\bf i}\,}}
\def\jj{\mbox{\,{\bf j}\,}}
\def\AA{{\cal A}}
\def\BB{{\cal B}}
\def\DD{{\cal D}}
\def\EE{{\mbox{\bf 1}}}
\def\OO{{\mbox{\bf 0}}}
\def\kk{{\mbox{\bf k}}}
\def\R{\mathbb{R}}
\def\F{\mathbb{F}{\ssc\,}}
%\def\similar{\rb{-4pt}{\mbox{\,\~\,}}}
\def\similar{\backsim}
\def\Llra{\Longleftrightarrow}
\def\Lra{\Longrightarrow}
\def\Lla{\Longleftarrow}
\def\mat#1#2{\mbox{$\left(\begin{array}{#1}#2\end{array}\right)$}}
\def\det#1#2{\mbox{$\left|\begin{array}{#1}#2\end{array}\right|$}}
\def\eset{\emptyset}
\parindent=5ex
\setlength{\parindent}{0pt}
\setlength{\parskip}{1ex plus 0.5ex minus 0.2ex}
\newtheorem{Example}{\text{例}}


\begin{document}
\begin{CJK*}{UTF8}{gbsn}
\title{Homework2}
\author{李青林(5110309074)}
\date{}
\maketitle

\section{Q1}
\textbf{Q1. Scoreboarding vs. Tomasulo’s Algorithm}\\

ADD.S F1, F2, F3\\
MUL.S F1, F2, F3\\

Scoreboarding would stop because of the WAW hazard.\\
But Tomasulo would continue would continue because the multiplier is free and it can avoid WAW hazard by register renaming.

\section{Q2}
\textbf{Q2. Branch Prediction}

\begin{enumerate}[a)]

\item 
m=2, n=2

\item
not taken\\

B1: not taken\\
B2: taken\\

The predictor for B3 should be $\mathbf{01}$\\
Since the state of predictor $\mathbf{01}$ is $\mathtt{00}$, it predicte not taken.

\item
$\mathbf{00}$
Because B3 is not taken

\item
Each entry contains $8$ bits.

$$10000/8=1250$$

The number of entry is $1024 = 2^{10}$
	
\end{enumerate}

\section{Q3}
\textbf{Q3. Code Scheduling}

\begin{enumerate}[a)]
\item
L.S F0, 0(R1)\\
L.S F1, 0(R2)\\
Stall\\
Stall\\
ADD.S F0, F0, F1\\
L.S F2, 0(R3)\\
L.S F3, 0(R4)\\
Stall\\
Stall\\
MUL.S F2, F2, F3\\
Stall\\
Stall\\
Stall\\
Stall\\
ADD.S F0, F0, F2\\
Stall\\
Stall\\
S.S F0, 0(R5)\\

18 cycles in all\\

\item
L.S F2, 0(R3)\\
L.S F3, 0(R4)\\
L.S F0, 0(R1)\\
L.S F1, 0(R2)\\
MUL.S F2, F2, F3\\
ADD.S F0, F0, F1\\
ADD.S F0, F0, F2\\
S.S F0, 0(R5)\\

L.S F2, 0(R3)\\
L.S F3, 0(R4)\\
L.S F0, 0(R1)\\
L.S F1, 0(R2)\\
MUL.S F2, F2, F3\\
Stall\\
ADD.S F0, F0, F1\\
Stall\\
Stall\\
ADD.S F0, F0, F2\\
Stall\\
Stall\\
S.S F0, 0(R5)\\

13 cycles in all\\

\end{enumerate}

\section{Q4}
\textbf{Q4. VLIW}

\begin{enumerate}[a)]
\item
$ $
\begin{table}[!hbp]
\resizebox{\textwidth}{!}{
\begin{tabular}{|c|c|c|c|c|c|}
\hline 
ALU1 & ALU2 & MU1 & MU2 & FADD & FMUL \\ 
\hline 
add r1, r1, -4 & add r2, r2, -4 & ld f1, 0(r1) & ld f1, 0(r1) &  &  \\ 
\hline 
add r3, r3, -4 & add r3, r4, -1 & ld f3, 0(r1) &  &  &  \\ 
\hline 
 &  &  &  &  &  \\ 
\hline 
 &  &  &  &  & fmul f4, f2, f1 \\ 
\hline 
 &  &  &  &  &  \\ 
\hline 
 &  &  &  &  &  \\ 
\hline 
 &  &  &  &  &  \\ 
\hline 
 &  &  & st f4, 4(r1) & fadd f5, f4, f3 &  \\ 
\hline 
 &  &  &  &  &  \\ 
\hline 
 &  &  &  &  &  \\ 
\hline 
 &  &  &  &  &  \\ 
\hline 
 & benz r4, loop & st f5, 0(r3) &  &  &  \\ 
\hline 
 &  &  &  &  &  \\ 
\hline 
 &  &  &  &  &  \\ 
\hline 
\end{tabular} 
}
\end{table}
\newpage
\item
$ $
\begin{table}[!hbp]
\resizebox{\textwidth}{!}{
\begin{tabular}{|c|c|c|c|c|c|}
\hline 
ALU1 & ALU2 & MU1 & MU2 & FADD & FMUL \\ 
\hline 
add r4, r4, -2 &  & ld f1, 0(r1) & ld f2, 0(r2) &  &  \\ 
\hline 
add r1, r1, -8 & add r2, r2, -8 & ld f6, -4(r1) & ld f7, -4(r2) &  &  \\ 
\hline 
add r3, r3, -8 &  & ld f3, 0(r3) & ld f8, -4(r3) &  &  \\ 
\hline 
 &  &  &  &  &  \\ 
\hline 
 &  &  &  &  & fmul f4, f2, f1 \\ 
\hline 
 &  &  &  &  & fmul f9, f8, f7 \\ 
\hline 
 &  &  &  &  &  \\ 
\hline 
 &  &  &  &  &  \\ 
\hline 
 &  &  & st f4, 8(r1) & fadd f5, f4, f3 &  \\ 
\hline 
 &  &  & st f9, 4(r1) & fadd f10, f9, f8 &  \\ 
\hline 
 &  &  &  &  &  \\ 
\hline 
 &  &  &  &  &  \\ 
\hline 
 &  &  & st f5, 8(r3) &  &  \\ 
\hline 
 & benz r4, loop &  & st f10, 4(r3) &  &  \\ 
\hline 
 &  &  &  &  &  \\ 
\hline 
\end{tabular} 
}
\end{table}

It do give us better performance.

\end{enumerate}

\section{Q5}
\textbf{Q5. Simple Cache}

\begin{enumerate}[a)]

\item
$ $\\
\begin{tabular}{|c|c|c|c|c|c|c|}
\hline 
reference	& 1 		& 4 		& 8 		&5& 20 	& 17 	 \\ 
\hline 
hit/miss		& miss 	& miss 	& miss 	&miss& miss 	& miss 	 \\ 
\hline 
reference &19& 56 & 9 & 11 & 4 & 43  \\ 
\hline 
hit/miss &miss& miss & miss & miss & miss & miss  \\ 
\hline
reference &5& 6 & 9 & 17 &&\\
\hline
hit/miss &hit& miss & hit &hit&&\\
\hline 
\end{tabular} 

\begin{tabular}{|c|c|c|c|c|c|}
\hline 
block NO. &0 & 1 & 2 & 3 & 4  \\ 
\hline 
content && 17 &  & 19 & 4  \\ 
\hline 
block NO. &5& 6 & 7 & 8 & 9  \\ 
\hline 
content &5& 6 &  & 56 & 9   \\ 
\hline 
block NO. &10& 11 & 12 & 13 & 14  \\ 
\hline 
content & & 43 &  &  &    \\ 
\hline
block NO. & 15 &&&&\\
\hline
content & &&&&\\
\hline 
\end{tabular} 

\newpage
\item
$ $\\

\begin{tabular}{|c|c|c|c|c|c|c|}
\hline 
reference	& 1 		& 4 		& 8 		&5& 20 	& 17 	 \\ 
\hline 
hit/miss		& miss 	& miss 	& miss 	&hit& miss 	& miss 	 \\ 
\hline 
reference &19& 56 & 9 & 11 & 4 & 43  \\ 
\hline 
hit/miss &hit& miss & miss & hit & miss & miss  \\ 
\hline
reference &5& 6 & 9 & 17 &&\\
\hline
hit/miss &hit& hit & miss &hit&&\\
\hline 
\end{tabular} 

\begin{tabular}{|c|c|c|c|c|}
\hline 
block1 & 1 & 2 & 3 & 4 \\ 
\hline 
content & 16 & 17 & 18 & 19 \\ 
\hline 
block2 & 1 & 2 & 3 & 4 \\ 
\hline 
content & 4 & 5 & 6 & 7 \\ 
\hline 
block3 & 1 & 2 & 3 & 4 \\ 
\hline 
content & 8 & 9 & 10 & 11 \\ 
\hline 
block4 & 1 & 2 & 3 & 4 \\ 
\hline 
content &  &  &  &  \\ 
\hline 
\end{tabular} 

\item
$ $\\
\begin{tabular}{|c|c|c|c|c|c|c|}
\hline 
reference	& 1 		& 4 		& 8 		&5& 20 	& 17 	 \\ 
\hline 
hit/miss		& miss 	& miss 	& miss 	&miss& miss 	& miss 	 \\ 
\hline 
reference &19& 56 & 9 & 11 & 4 & 43  \\ 
\hline 
hit/miss &miss& miss & miss & miss & hit & miss  \\ 
\hline
reference &5& 6 & 9 & 17 &&\\
\hline
hit/miss &hit& miss & hit &hit&&\\
\hline 
\end{tabular} 

\begin{tabular}{|c|c|c|}
\hline
NO. & way0 & way1\\
\hline
0 & 56 & 8  \\
\hline
1 & 17 & 9  \\
\hline
2 &  &   \\
\hline
3 & 43 & 11  \\
\hline
4 & 4 & 20  \\
\hline
5 & 5 &   \\
\hline
6 & 6 &   \\
\hline
7 &  &   \\
\hline
\end{tabular}
\newpage
\item
$ $\\

\begin{tabular}{|c|c|c|c|c|c|c|}
\hline 
reference	& 1 		& 4 		& 8 		&5& 20 	& 17 	 \\ 
\hline 
hit/miss		& miss 	& miss 	& miss 	&miss& miss 	& miss 	 \\ 
\hline 
reference &19& 56 & 9 & 11 & 4 & 43  \\ 
\hline 
hit/miss &miss& miss & miss & miss & hit & miss  \\ 
\hline
reference &5& 6 & 9 & 17 &&\\
\hline
hit/miss &hit& miss & hit &hit&&\\
\hline 
\end{tabular} 

\begin{tabular}{|c|c|c|c|c|}
\hline
NO. & 0 & 1 & 2 & 3 \\ 
\hline
content & 17 & 9 & 6 & 5 \\
\hline
NO. & 4 & 5 & 6 & 7\\
\hline
content & 43 & 4 & 11 & 56\\
\hline
NO. & 8 & 9 & 10 & 11\\
\hline
content & 19 & 20 & 8 & 1\\
\hline
NO. & 12&13&14&15\\
\hline
content &&&&\\
\hline
\end{tabular}

\end{enumerate}

\section{Q6}
\textbf{Q6. SimpleScalar Assignment}

anagram:\\
\begin{tabular}{|c|c|}
\hline 
  & hit rate \\ 
\hline 
1bit-GAp & 0.9671 \\ 
\hline 
1bit-PAg & 0.9684 \\ 
\hline 
1bit-bimod & 0.9335 \\ 
\hline 
2bit-GAp & 0.9766 \\ 
\hline 
2bit-PAg & 0.9750 \\ 
\hline 
2bit-bimod & 0.9614 \\ 
\hline 
\end{tabular} 

cc1:\\
\begin{tabular}{|c|c|}
\hline 
 & hit rate \\ 
\hline 
1bit-GAp & 0.8557 \\ 
\hline 
1bit-PAg & 0.8210 \\ 
\hline 
1bit-bimod & 0.8641 \\ 
\hline 
2bit-GAp & 0.8454 \\ 
\hline 
2bit-PAg & 0.8191 \\ 
\hline 
2bit-bimod & 0.8823 \\ 
\hline 
\end{tabular} 

go:\\
\begin{tabular}{|c|c|}
\hline 
 & hit rate \\ 
\hline 
1bit-GAp & 0.7419 \\ 
\hline 
1bit-PAg & 0.7075 \\ 
\hline 
1bit-bimod & 0.7361 \\ 
\hline 
2bit-GAp & 0.7305 \\ 
\hline 
2bit-PAg & 0.7156 \\ 
\hline 
2bit-bimod & 0.7023 \\ 
\hline 
\end{tabular} 
\end{CJK*}
\end{document}

