%%%%%%%%%%%%%%%%%%%%%%%%%%%%%%%%%%%%%%%%%%%%%%%%%%%%%%%%%%%%%%%%%%%%%%%%
\documentclass[12pt]{article}
\usepackage{amssymb}
\usepackage{amsmath,amssymb,CJK}
\usepackage{graphicx}
\usepackage{subfigure}
\usepackage{listings}
\usepackage{enumerate}

\openup 7pt\pagestyle{plain} \topmargin -40pt \textwidth
14.5cm\textheight 21.5cm
\parskip .09 truein
\baselineskip 4pt\lineskip 4pt \setcounter{page}{1}
\def\a{\alpha}\def\b{\beta}\def\d{\delta}\def\D{\Delta}\def\fs{\footnotesize}
\def\g{\gamma}
\def\G{\Gamma}\def\l{\lambda}\def\L{\Lambda}\def\o{\omiga}\def\p{\psi}
\def\se{\subseteq}\def\seq{\subseteq}\def\Si{\Sigma}\def\si{\sigma}\def\vp{\varphi}\def\es{\varepsilon}
\def\sc{\scriptstyle}\def\ssc{\scriptscriptstyle}\def\dis{\displaystyle}
\def\cl{\centerline}\def\ll{\leftline}\def\rl{\rightline}\def\nl{\newline}
\def\ol{\overline}\def\ul{\underline}\def\wt{\widetilde}\def\wh{\widehat}
\def\rar{\rightarrow}\def\Rar{\Rightarrow}\def\lar{\leftarrow}
\def\Lar{\Leftarrow}\def\Rla{\rightleftarrow}\def\bs{\backslash}
\def\ra{\rangle}\def\la{\langle}\def\hs{\hspace*}\def\rb{\raisebox}
\def\ni{\noindent}\def\hi{\hangindent}\def\ha{\hangafter}
\def\vs{\vspace*}
\def\hom#1{\mbox{\rm Hom($#1,#1$)}}
\def\thebeg{\vskip8pt\par\ni}
\def\theend{\vskip5pt\par}
\def\chabeg{\pagebreak\par}
\def\chaend{\vskip20pt\par}
\def\secbeg{\vskip15pt\par}
\def\secend{\vskip10pt\par}
\def\exebeg{\vskip10pt}
\def\exeend{\vskip6pt}
\def\undot#1{\mbox{$\mbox{#1}\hs{-1.5ex}_{_{\dis\centerdot}}\,\,$}}
\def\qed{\hfill\mbox{$\Box$}}
\def\C{\mathbb{C}}
\def\Q{\mathbb{Q}}
\def\ii{\mbox{\,{\bf i}\,}}
\def\jj{\mbox{\,{\bf j}\,}}
\def\AA{{\cal A}}
\def\BB{{\cal B}}
\def\DD{{\cal D}}
\def\EE{{\mbox{\bf 1}}}
\def\OO{{\mbox{\bf 0}}}
\def\kk{{\mbox{\bf k}}}
\def\R{\mathbb{R}}
\def\F{\mathbb{F}{\ssc\,}}
%\def\similar{\rb{-4pt}{\mbox{\,\~\,}}}
\def\similar{\backsim}
\def\Llra{\Longleftrightarrow}
\def\Lra{\Longrightarrow}
\def\Lla{\Longleftarrow}
\def\mat#1#2{\mbox{$\left(\begin{array}{#1}#2\end{array}\right)$}}
\def\det#1#2{\mbox{$\left|\begin{array}{#1}#2\end{array}\right|$}}
\def\eset{\emptyset}
\parindent=5ex
\setlength{\parindent}{0pt}
\setlength{\parskip}{1ex plus 0.5ex minus 0.2ex}
\newtheorem{Example}{\text{例}}
\begin{CJK*}{UTF8}{gbsn}

\date{}
\title{Homework3}
\author{Qinglin Li, 5110309074}
\begin{document}
\maketitle
\section*{Q1. Cache}
	\subsection*{i}
	The query sequence:

	1,2,3,4,5,6,7,8,9,1,2,3,10,11,1,2,3,12,13,14,15,16,17,10,11\\

	8 frames:\\
	\begin{tabular}{|c|c|c|}
	\hline
	query & cache & miss/hit \\
	\hline

	1..8 & 1,2,3,4,5,6,7,8 & miss \\ \hline

	9 & 2,3,4,5,6,7,8,9 & miss \\ \hline

	1 & 3,4,5,6,7,8,9,1 & miss \\ \hline

	2 & 4,5,6,7,8,9,1,2 & miss \\ \hline

	3 & 5,6,7,8,9,1,2,3 & miss \\ \hline

	10 & 6,7,8,9,1,2,3,10 & miss \\ \hline

	11 & 7,8,9,1,2,3,10,11 & miss \\ \hline

	1 & 7,8,9,1,2,3,10,11 & hit \\ \hline

	2 & 7,8,9,1,2,3,10,11 & hit \\ \hline

	3 & 7,8,9,1,2,3,10,11 & hit \\ \hline

	12..17 & 10,11,12,13,14,15,16,17 & miss \\ \hline

	10 & 10,11,12,13,14,15,16,17 & hit \\ \hline

	11 & 10,11,12,13,14,15,16,17 & hit \\ \hline

	\end{tabular}

	20 misses, 5 hits\\
	\newpage
	9 frames:\\
	\begin{tabular}{|c|c|c|}
	\hline
	query & cache & miss/hit \\
	\hline

	1..9 & 1,2,3,4,5,6,7,8,9 & miss \\ \hline

	1 & 1,2,3,4,5,6,7,8,9 & hit \\ \hline

	2 & 1,2,3,4,5,6,7,8,9 & hit \\ \hline

	3 & 1,2,3,4,5,6,7,8,9 & hit \\ \hline

	10 & 2,3,4,5,6,7,8,9,10 & miss \\ \hline

	11 & 3,4,5,6,7,8,9,10,11 & miss \\ \hline

	1 & 4,5,6,7,8,9,10,11,1 & miss \\ \hline

	2 & 5,6,7,8,9,10,11,1,2 & miss \\ \hline

	3 & 6,7,8,9,10,11,1,2,3 & miss \\ \hline

	12..17 & 1,2,3,12,13,14,15,16,17 & miss \\ \hline

	10 & 2,3,12,13,14,15,16,17,10 & miss \\ \hline

	11 & 3,12,13,14,15,16,17,10,11 & miss \\ \hline

	\end{tabular}

	22 misses, 3 hits
	\subsection*{ii}
	No.\\
	With LRU, 9 frames cache would contain whatever contained in 8 frames cache


\section*{Q2.Memory}
	\subsection*{i}
	$$(M+2D+1)\times d\%\times(1-H3)\times H2\times w\%$$
	
	\subsection*{ii}
	zoom in please	
	
	\includegraphics[scale=0.37]{tree}

\section*{Q3.Hard Disk}
	\subsection*{i}
	$$20000\times5000\times512~\text{bytes}=333.8\text{GB}$$
	
	\subsection*{ii}
	All the other disks need to be read and no other disks need to be written.
	
	\subsection*{iii}
	The write request of block 12, 23 and 66 can be paralleled.\\
	Write request of block 6 would write P0 on disk 7.\\
	Write request of block 27 would write disk 7.\\
	Write request of block 28 would write disk 0.\\
	Write request of block 50 would write P0 on disk 0.\\
	
\section*{Q4.}
	A=17bits\\
	B=8bits\\
	C=7bits\\
	D=1024bits\\
	E=17bits\\
	G=13bits\\
	H=7bits\\
	I=12bits\\
	J=17bits\\
\end{CJK*}
\end{document}
